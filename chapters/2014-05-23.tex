\documentclass[../main/Notes.tex]{subfiles}
\begin{document}

\section[Bayesian Inference Examples]{Bayesian Inference Examples \iftoggle{showdates}{\small{\textit{2014-05-23}}}{}}
\label{ex4_4_solution}
This section is basically about Exercise 4 on the 4th Tutorial Sheet (see page \pageref{exsheet4}). The idea is that we have a multiple choice test where the answers are not simply true or false but can be any value between 0 and 1 representing your belief in this statement to be true (where 0 corresponds to your belief in this statement being false, 1 that you belief it's true). What does a proper scoring do in this example? What is calibration?\index{Calibration}

\subsection{Honesty}
If we use a proper scoring rule\index{Proper Scoring Rule} the participant can minimize her error if she always states her honest belief in the statement. It is obvious that this does not help in getting any points for statements where you have no clue about its real truth value and you can still get lucky if you gamble and just guess a value for a statement. 

But for a huge number of questions it is highly unlikely that you gain anything and we proved in the other exercises that it is optimal to state your true belief.

\subsection{Calibration}\index{Calibration|(}
If you are well-calibrated then $80 \%$ of the statements you marked with $80 \%$ should be true. Calibration is the bridge between frequentist and Bayesian view on this topic. Calibration can only be measured if you have a huge enough sample space, which is not often the case since you have rather twenty than two hundred questions. 

Afterwards it is not possible (in our setup) to tell whether wrong answers are due to lying or bad calibration. There is another problem: It is very hard for a normal person to be well calibrated (even if they try). Psychological studies show that most people systematically overestimate their own belief. Meaning that they would write $1$ where their true belief is rather $0.92$. 

People also overestimate small probabilities. For example, people think that it is rather likely to die in a plane crash than in a car accident, although this is rather unlikely compared to car accidents. One could take such studies into account and try to come up with correction terms for the common failures, but this is rather complicated. This is part of the reason why these proper scoring or multiple choice tests of this form are not widely used. The other reason is that not many people know about this stuff and the effect is not that great to even start with all the trouble.
\index{Calibration|)}

\end{document}