\documentclass[../main/Notes.tex]{subfiles}
\begin{document}

\section[Solution 8: Choice Models I+II]{Solution 8: Choice Models I+II \iftoggle{showdates}{\small{\textit{2014-07-11}}}{}}\index{Choice models}

\subsection*{Exercise 1}\index{Strong Stochastic Transitivity}
\texttt{MATLAB} code for exercise 1:
\matlabcode{../data/sheet8_pce.m}

This code simply initializes the three $\mu$ and creates the distance matrix between them. Then it simply calculates the probability for each event (i.e. how high is the probability that a subject chooses $1$ over $2$?). The probability for the case that a subject chooses 2 over 1, 3 over 2, and 1 over 3 is then just the multiplication of these three individual events: $p = P(2,1) \cdot P(3,2) \cdot P(1,3) \approx 0.07$.

\bigskip

So now we know the probability for one single subject. How is the probability for $n$ subjects?

To answer that question we continue our script from before and add a simulation for the case that 2 is chosen over 1, 3 is chosen over 2, and 1 is chosen over three, to see how often the strong stochastic transitivity gets violated.

\bigskip

\textbf{Reminder:} Strong Stochastic Transitivity
\begin{align*}
\text{\texttt{if }}   & p_{jk} \geq 0.5\ \&\ p_{ij} \geq 0.5 \\
\text{\texttt{then }} & p_{ik} \geq \max{\left(p_{jk},p_{ij}\right)}
\end{align*}

The following \texttt{MATLAB} code simulates the choices with the probabilities taken from the code above (Matrix $P$) for $m$ subjects and counts the violations of strong stochastic transitivity. In the end it just divides the number of violations by the number of simulated trials to come up with a probability.
\matlabcode{../data/sheet8_strong_stochastic_transitivity.m}

\sidenote{Not directly covered in the lecture.}
We can also invert this process (at least parts of it). Assuming we have the $q$ values for a matrix (e.g. by asking random people whether they prefer pizza tonno over pizza salami, pizza tonno over pizza margarita, etc.) we can calculate the $\mu$s, i.e. the utilities for those choices. However, we are not able to find the x-shift, so if we have two $\mu$s, e.g. $\mu_1 = 1.5$ and $\mu_2 = 2$, the result of our calculations will turn out to be $\mu_1=0$ and $\mu_2=0.5$. An example \texttt{MATLAB} code can be found in the appendix on page \pageref{app:matlabcode_ex8_1}\label{back:matlabcode_ex8_1}.



\subsection*{Exercises 2-4}
The commented \texttt{MATLAB} code which was provided as a solution can be found in the appendix on page \pageref{app:matlabcode_ex8_2_show}\label{back:matlabcode_ex8_2_show}.

It employs the functions \texttt{thurstone} (page \pageref{app:matlabcode_ex8_2_thurstone}\label{back:matlabcode_ex8_2_thurstone}) and \texttt{restle} (page \pageref{app:matlabcode_ex8_2_restle}\label{back:matlabcode_ex8_2_restle}) which are solutions to the exercises two to four.

The \texttt{choiceplot} function (page \pageref{app:matlabcode_ex8_2_choiceplot}\label{back:matlabcode_ex8_2_choiceplot}) makes a plot which shows how good a model fits the data. The more data points are inside the red borders, the better the data is fitted by the model.

Please refer to the respective pages in the appendix for further information.

\end{document}
