\documentclass{article}
\usepackage{amsmath}
\usepackage{tikz}
\usetikzlibrary{trees}

\begin{document}

\LARGE PMPC Tutorial Sheet 1\\
\small Lisa Goerke, Kai Standvoss, Andrea Suckro, Sebastian H\"offner
\vspace{0.5cm}
\normalsize


\section*{Exercise 2}
\subsection*{Exercise 2.a}

We use the formula for calculating odds:
\begin{align}
\frac{p}{1-p} = \frac{s_2}{s_1}
\end{align}
The probability for rolling a 6 is $\frac{1}{6}$, $s_1$ can be set to 10.
\begin{align}
\frac{\frac{1}{6}}{1-\frac{1}{6}} = \frac{s_2}{10} \Leftrightarrow s_2 = 2
\end{align}
So our stack should be \$2.


\subsection*{Exercise 2.b}
We use the same formula again, this time with $s_1=25$.
\begin{align}
\frac{\frac{1}{6}}{1-\frac{1}{6}} = \frac{s_2}{25} \Leftrightarrow s_2 = 5
\end{align}
So we accept all stakes $s_2 \geq \$ 5$.


\subsection*{Exercise 2.c}
We use the formula for odds and solve it for $p$.
\begin{align}
                & & \frac{p}{1 - p} & = \frac{s_2}{s_1}                     & & \\
\Leftrightarrow & & p               & = \frac{s_2}{s_1} (1 - p)             & & \\
\Leftrightarrow & & p               & = \frac{s_2}{s_1} - \frac{s_2}{s_1} p & & \\
\Leftrightarrow & & p s_1           & = s_2 - p s_2                         & & \\
\Leftrightarrow & & p s_1 + p s_2   & = s_2                                 & & \\
\Leftrightarrow & & p (s_1 + s_2)   & = s_2                                 & & \\
\Leftrightarrow & & p               & = \frac{s_2}{s_1 + s_2}               & &
\end{align}
So the probability should be $p = \frac{s_2}{s_1 + s_2}$.


\section*{Exercise 3}
The average amount of juice over all trials shall be 1ml. The formula for this is the known:
\begin{align}
EV = s_1 p_1 + s_2 p_2
\end{align}
Where $p_1$ is the probability for a red trial and $p_2$ for a blue trial. Since we always have either a red or a blue trial, we get the following:
\begin{align}
EV = 1 \  ml &= s_1 p_1 + s_2 (1 - p_1)\\
\Leftrightarrow s_1 &= \frac{1 \  ml - s_2 (1 - p_1)}{p_1}\\
\Leftrightarrow s_2 &= \frac{1 \  ml - s_1 p_1}{1 - p_1}
\end{align}
In order to obtain concrete values we need to fix $p_1$ and one of the stakes (either $s_1$ or $s_2$) with a value $\leq 1 \  ml$.


\section*{Exercise 4}
In general the statements with an conjunction are less likely. So a possible order could be:\\
$(c)\Rightarrow(d)\Rightarrow(e)\Rightarrow(f)\Rightarrow(a)\Rightarrow(b)\Rightarrow(g)$


\section*{Exercise 5}
\input{"Tutorial Sheet 1.5.tex"}



\end{document}