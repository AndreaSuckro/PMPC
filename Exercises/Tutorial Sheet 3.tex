\def \TutorialSheetNumber{3}
\documentclass{article}

\usepackage{amsmath}

\usepackage{eurosym}
\DeclareRobustCommand{\officialeuro}{
  \ifmmode\expandafter\text\fi
  {\fontencoding{U}\fontfamily{eurosym}\selectfont e}}

\usepackage{tikz}
\usetikzlibrary{trees}

\setlength\parindent{0pt}

\begin{document}

\begin{center}
\LARGE PMPC Tutorial Sheet \TutorialSheetNumber\\
\small Lisa Goerke, Kai Standvoss, Andrea Suckro, Sebastian H\"offner
\end{center}
\vspace{0.5cm}
\normalsize


\section*{Exercise 2}
Before we can calculate any probabilities, due to many unclearities in the excercise, we make some assumptions about the circumstances of the case. First we approximate the number of subjects taking part in the DNA test still to be 100.000 even though probably this rate would be a lot less, but thereby we at least get an upper bound for our probabilities. Further we assume that the murderer really is one of the men which renders the probability of being guilty as \\ 
$P(Guilty) = \frac{1}{100,000}$\\
and therefore\\
$P(\neg G) = \frac{99,999}{100,000}$\\
We are further given the probability of having a positiv test result given you are guilty \\
$P(Positiv|G) = 0.9999 $ \\
The actual probability we are interested in is the one of really being guilty given a positiv test result:
\begin{align*}
  P(G | P) & = \frac{P(P|G) \cdot P(G)}{P(P)} \\
  P(G | P) & = \frac{P(P|G) \cdot P(G)}{\sum\limits_{i=1}^{N}(P(P|G_i)\cdot P(G_i)}\\
  P(G | P) & = \frac{P(P|G) \cdot P(G)}{P(P|G)\cdot P(G) + P(P|\neg G)\cdot P(\neg G)} \\
  P(G | P) & = \frac{0.9999 \cdot \frac{1}{100,000}}{0.9999 \cdot \frac{1}{100,000} + 0.00001 \cdot \frac{99,999}{100,000}}\\
  P(G | P) & = 0.49998
\end{align*}
 That means the actual probability that the accused person really is guilty is just about 50\%. Again neglecting the high base rate, at first glance in court  it may seem very likely that the defendant is the murderer but we as clever attornies, knowing our statistics, can argue that with 100,000 participants taking part in the DNA test, the chance for a false alarm is still pretty high even though the test is so accurate. Of cause the probability of 50\% is just an upper bound and can be much less for fewer participants but nevertheless shows the huge mistake one makes by neglecting the effect of a high base rate.  
\section*{Exercise 4}
The odds are $7 : 2$ for Portugal to win, it means that if Adam bets $10\euro$ on Portugal the bookmaker bets $2,857€$ on Greece. In case Portugal wins Adam will get $12,86€$, which is a gain of $2,86€$.
Boris bets $10€$ on Greece and the odds are 2:5 that Greece wins. The bookmaker bets $25€$ on Portugal, so Boris will get $35€$ if Greece wins, which is a gain of $25€$.

\section*{Exercise 5}
Charly bets $6,70$ in total.
If the odds were $7 : 2$ and $2 : 7$, Charly would lose $0.40€$ if Portugal wins, he would win $1.40$ if Greece wins. The bookmaker however wants to make money, that's why he fixes the odd in a way that he will earn some.\\
$7 : 2$ for Portugal $\rightarrow 4,90€ : 1,40€$ $\rightarrow$ Charly gets $6,30€$, he loses $0.40$.\\
$2 : 5$ for Greece $\rightarrow 1,80€ : 4,50€$ $\rightarrow$ Charly gets $6,30€$, he loses $0.40$.

\section*{Exercise 9}
Before we can play the game at all we have to pay 1000 Euros. In order to know for which N we make money we need to know when $2^{N}$ gets bigger than 1000. Since $2^{10} = 1024$ we begin to make money in the $10^{th}$ toss.
The probability for tail to come is $\frac{1}{2}$ and because we needed tail in the first 9 trials in order to win the probability for that is $\frac{1}{2}^{9} = 0.002$ and thus the probability that we lose is $1-0.002 = 0.998 = 99.8\%$ \\
Expected value = $-0.998\cdot 1000 + 0.002\cdot 2^{N}$ 

\end{document}
