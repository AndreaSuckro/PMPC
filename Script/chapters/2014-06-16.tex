\documentclass[../main/Notes.tex]{subfiles}
\begin{document}

\section[Signal Detection Theory II]{Signal Detection Theory II \iftoggle{showdates}{\small{\textit{2014-06-16}}}{}}

\subsection{Objective Sensitivity}
In the previous lecture we saw how ROC curves helped us to measure the sensitivity of a subject in a decision task. We could compare the curves for several subjects and find out which one has the 'better' sensitivity. But it would be better to have a single value to compare the sensitivity. For the case where signal and noise are gaussians with same variance this value is the SNR(Signal to Noise Ratio).

FANCY PLOT HIER

Let's calculate that!

We subtract now the mean of the No-responses. 
\begin{align*}
p(H) = \int_{0}^{\infty} \varphi \left( x,\mu_y,\delta \right)dx &= 1 - \Phi\left(\theta,x_y,\delta\right)\\
                                                                 &= 1 - \Phi\left(\theta-\mu_n,\mu_y-\mu_n,\delta \right)\\ 
p(FA) = \int_{\theta}^{\infty} \varphi \left( x,\mu_n,\delta \right)dx &= 1 - \Phi\left(\theta,x_n,\delta\right)\\
                                                                       &= 1 - \Phi\left(\theta-\mu_n,0,\delta \right)   
\end{align*}
We may also adapt no the variance by dividing through $\delta$.
\begin{align*}
p(H) = 1 - \Phi\left(\frac{\theta-\mu_n}{\delta},\frac{\mu_y-\mu_n}{\delta},1 \right) &= 1 - \Phi\left(\theta'-d',0,1\right)\\
                                                                                      &= 1 - \Phi\left(\theta'-d'\right)\\
p(FA) = 1 - \Phi\left(\frac{\theta-\mu_n}{\delta},0,1 \right) &= 1 - \Phi\left(\theta' \right) 
\end{align*}
We can rearrange the last formular to:
\begin{align*}
\Phi\left(\theta'\right) &= 1 - P(FA)\\
\theta'&=\Phi^{-1}\left(1-P(FA)\right)\\
&=-\Phi^{-1}\left(P(FA)\right)
\end{align*}
The last step is possible because the gaussian is symmetric. Now we try to find a formula for $d'$ as well. 
\begin{align*}
\Phi\left(\theta'-d'\right) &= 1 - P(H)\\
\theta'-d'&=\Phi^{-1}\left(1-P(H)\right)\\
d' &= \theta'+\Phi^{-1}\left(P(H)\right)\\
   &= \Phi^{-1}\left(P(H)\right)-\Phi^{-1}\left(P(FA)\right)
\end{align*}
By this we disentangled the sensitivity and the response bias of the subject. $\theta$ is rather a bias than a threshold.

\subsection{Is there a sensory threshold?}


\end{document}
