\section[Frequentist Inference Examples]{Frequentist Inference Examples \iftoggle{showdates}{\small{\textit{2014-05-26}}}{}}

\subsection{Bayesian Inference for Thumbtacks}
We return to the example from the first lecture (see page \pageref{example:Thumbtack Toss}). Let us imagine we have thrown a thumbtack $n$ times. The series of data we get from that may be: 01110101101, a binary string consisting of $n$ entries (0 encodes tail, 1 head). Let $x_1....x_n =: X$ be a random variable. What is the probability of the data given our vague belief about q (for a coin we have the strong intuition of believing that $q=0.5$, but for the thumbtack we are not sure)? The probability of the whole data ($X$) is the product of the probability of all single events ($x_i$):
\begin{align*}
P(X|q) = \prod\limits_{i=1}^n{q^{x_i}\left(1-q\right)^{1-x_i}}=\underbrace{q^h\left(1-q\right)^t}_\text{Bernoulli}
\end{align*}
h: \#heads
t: \#tails

But since we do not know $q$ we would also like to find the best $q$ that explains the observed data. In other words: what is the probability of a specific $q$ given the data? This can again be expressed with Bayes' Rule\index{Bayes' Rule}:

\begin{align*}
\overbrace{p(q|X)}^\text{posterior} = \frac{P(X|q)\overbrace{p(q)}^\text{prior}}{\int_0^1 P(X|q)p(q) dq}
\end{align*}

The question arising here is what shall we take as the prior? In principle it is just our personal belief about the thumbtack, one experimenter might believe it is $0.7$, others believe different values. This \textit{subjectivity} troubles many Non-Bayesians. The good thing is, that it does not really matter which prior you choose, if you have enough data the result will still converge to the real $p(q|X)$!

Since we have no concrete clue and it is not that important anyway, we may choose a distribution by pure convenience for $p(q)$. The distribution is called \textit{$\beta$ Distribution}\index{$\beta$-Distribution@B} :
\begin{align*}
p(q) &= \frac{q^{\alpha-1}(1-q)^{\beta-1}}{\int_0^1 q^{\alpha-1}(1-q)^{\beta-1}dq} = \frac{q^{\alpha-1}(1-q)^{\beta-1}}{B(\alpha,\beta)}
\end{align*}
We can now neglect the normalization term in the original $p(q|X)$. So we get rid of the integral in the denominator, since it is independent of $q$.
\begin{align*}
p(q|X) &\propto \underbrace{q^h(1-q)^t}_{P(X|q)} \underbrace{q^{\alpha-1}(1-q)^{\beta-1}}_{p(q)} = \underbrace{q^{h+\alpha-1}(1-q)^{t+\beta-1}}_\text{new $\beta$-distribution}
\end{align*}
If we now choose $\alpha_n = h+\alpha, \beta_n = t+\beta$, we can express the new $\beta$ Distribution as:
\begin{align*}
p(q|X) &= \frac{q^{\alpha_n-1}(1-q)^{\beta_n-1}}{B(\alpha_n, \beta_n)} 
\end{align*}

If the prior and posterior have the same distribution (like in this case) it is called \textit{conjugate}\index{Conjugation}.

\subsection{Map estimate (maximum a posteriori)}
In order to find the maximum posteriori term we need to calculate the first derivative of $p(q|X)$. That seems quite hard but we can reduce the problem by ignoring the normalization term $\frac{1}{B(\alpha_n, \beta_n)}$ (since it is independent from $q$) and taking the logarithm of the numerator, since this does not change the location of any maxima. By this we just have to maximize $\log(q^{\alpha_n-1}(1-q)^\beta_n-1)$.

\begin{align*}
\log{\left(q^{\alpha_n-1}(1-q)^{\beta_n-1}\right)} &= (\alpha_n-1)\log \est{q} + (\beta_n-1)\log (1-\est{q}) \\
\frac{\partial \left(\left(\alpha_n-1\right)\log \est{q} + \left(\beta_n-1\right)\log \left(1-\est{q}\right)\right)}{\partial \est{q}} &= \frac{\alpha_n-1}{\est{q}} - \frac{\beta_n-1}{1-\est{q}} = 0 \\
\frac{\alpha_n-1}{\est{q}} = \frac{\beta_n-1}{1-\est{q}} &\Leftrightarrow \frac{1-\est{q}}{\est{q}} = \frac{\beta_n-1}{\alpha_n-1} \\
\frac{1}{\est{q}} = \frac{\beta_n-1}{\alpha_n-1}+1 &= \frac{\beta_n-1}{\alpha_n-1} + \frac{\alpha_n-1}{\alpha_n-1} = \frac{\beta_n+\alpha_n-2}{\alpha_n-1} \\
\est{q} = \frac{\alpha_n-1}{\beta_n+\alpha_n-2} &= \frac{\alpha+h-1}{\alpha+\beta+h+t-2}
\end{align*}

\sidenote[.35]{The approach of using Bayes\-ian statistics with a prior that does not assume or put in any information is called 'Objective Bayes'. It is somehow the middle ground between the two opposing camps.}

For $\alpha = \beta = 1$ we get what we have already suspected before: $\est{q}=\frac{h}{h+t}$. $\alpha$ and $\beta$ are also called pseudo-counters. They represent data points you have not seen but believe to be realistic. This is a way to put your prior belief about the problem in the model - but it is also dangerous. If you have a strong belief in a hypothesis it will need more and more data to prove in the limit that you are wrong.

\subsection[Null Hypothesis Significance Testing (NHST)]{NHST Null Hypothesis Significance Testing}
In the previous section we examined how Bayesian people tackle the problem of finding a good model for a problem. For frequentists it is a bit more complicated. Remember that probabilities (like the probability for heads for a fair coin) are objective/fixed properties of the object. Writing $p(q)$ (as well as $p(q|X) \propto p(q)p(X|q)$) makes no sense for this reason. $q$ is not a random variable but a property and we need to find out its concrete value.

\subsubsection*{Experiment: Can someone discriminate between coke and pepsi?}

We would like to know what $p(q)$ is. But as frequentists we can't do that. So we start with the null hypothesis $H_0$: Subjects can't discriminate: $q = \frac{1}{2}, n = 25$. Where $q$ denotes the discrimination factor for the subjects and $n$ is the number of trials. We now measure H (\# ``Heads'': correct discriminations).

\begin{align*}
P(H=h|q) = \underbrace{ \binom{n}{h} \overbrace{q^h (1-q)^{n-h}}^\text{Bernoulli} }_\text{Binomial distribution}\index{Binomial Distribution}
\end{align*}
Now we introduce a criterion: subjects can discriminate if they get >20 correct answers. In that case you reject the null hypothesis.

--- some gauss distribution from notes

\begin{align*}
&P\left(H=h |q=\frac{1}{2},n=25\right) \\
&P\left(H>20 |q=\frac{1}{2},n=25\right) \mbox{ (p-Value)}
\end{align*}

$\alpha$ is the signal level $\rightarrow$ type I error rate that's acceptable (usually $5\%$). This is the probability that you say there is an effect even if there is none.

say $q = \frac{4}{5}$

--- another plot from notes

type II error $\beta \approx \frac{3}{4}$

tradeoff between $\alpha$ and $\beta$: ``easier'' for $q \rightarrow \frac{1}{2}$ for $n$ big: the power is $1-\beta$.