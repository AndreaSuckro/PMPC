\section[Measuring Beliefs II]{Measuring Beliefs II \iftoggle{showdates}{\small{\textit{2014-05-09}}}{}}

\subsection{Probabilities of Continuous Random Variables}

``How tall am I?'' -> 1.80, 1.70, 1.68, 1.69, 1.7034241?

Bayesian view would also ask: ``What do you think is the probability for that size?''

But continuous random variables are difficult: Infinite uncountable range of numbers, how to assign probabilities?
What's the probability of a real number?

\subsubsection[Probability Density Function (PDF)]{Solution 1: histograms}
Histogram! -> Limit of the histogram: Probability density function (PDF)
Make discrete space: 150-160, 160-170, etc.
Assign probabilities to those bins.
Change resolution (150-155, 155-160), reassign/reorder 

this is the density (and it's = 1 (area/integral))

two weird extreme cases:
- reassigning/reordering causes high densities
- equally distributed density causes probability of 0 for each exact number
   - however, you can calculate the probability for intervals with the integral
   -> you only want to bet on intervals

\subsubsection[Cumulative Density Function (CDF)]{Solution 2: Cumulative Density/Distribution Function (CDF)}
CDF is the integral of the PDF
 % rest see paper notes

\subsubsection{Solution 3: Parametric Distribution}
We use the Gaussian distribution

\begin{align*}
p(X=x) &= \frac{1}{2\pi\sigma^2}e^{-\frac{1}{2}\left(\frac{\mu-x}{\sigma}\right)^2} = \phi(x;\mu,\sigma) = \phi\left(\frac{\mu-x}{\sigma};0,1\right) \\
\frac{\mu-x}{\sigma}&=z \\
P(X \leq t) &= \int\limits_{-\infty}^{t}{p(X=x)} dx = \Phi(t;\mu,\sigma) \\
\mbox{probability of the standard deviation: } \\
P(\mu\sigma \leq X \leq \mu + \sigma) &= \Phi(\mu+\sigma;\mu,\sigma) - \Phi(\mu-\sigma;\mu,\sigma) \approx 68\%
\end{align*}

\begin{itemize}
	\item $\mu \pm \sigma \approx 68\%$
	\item $\mu \pm 2\sigma \approx 95\%$
	\item $\mu \pm 3\sigma \approx 99\%$
\end{itemize}

\subsection{Proper Scoring Rules}
Multiple choice test: better to say: ``How is your belief that this is right''

Example: EU pop > US pop? X = 1 if true, X = 0 if false

aim: high gain if true and high q, no gain if high q but false etc.

q is what you say your belief is 
\begin{align*}
L(X,q) &= (X-q)^2 \\
       &= X^2-2qX+q^2 \mbox{ note: $X^2 = X$, since only 1 or 2} \\
       &= X(1-2q)+q^2
\end{align*}

You lie: Your true belief is p, but you say q.

For best performance on test minimize expected loss $E(L(X,q)) = p(1-2q)+q^2$

p is ``a bet on $p\cdot num$ of questions are true'', so that's why the expected loss is like this

Minimize E: $1^{st}$ derivative:
\begin{align*}
E(L(X,q)) &= p(1-2q)+q^2 \\
\frac{\partial E}{\partial q}E(L(X,q)) &= -2p + 2q \\
\Leftrightarrow p = q
\end{align*}





