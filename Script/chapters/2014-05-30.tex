\documentclass[../main/Notes.tex]{subfiles}
\begin{document}

\section[Solution 5: Bayesian and Frequentist Inference I]{Solution 5: Bayesian and Frequentist Inference I \iftoggle{showdates}{\small{\textit{2014-05-30}}}{}}

\subsection*{Exercise 1}
\begin{figure}[!ht]
\centering
\begin{tikzpicture}
  \begin{axis}[axis x line*=bottom, axis y line*=left, enlargelimits=upper, axis on top, domain=0:1, legend style={at={(0.98,0.5)}, anchor=east}, x post scale=2, y post scale=2]
  \addplot[smooth, samples=100, color=blue]    {betaDist(1,1)};
    \addlegendentry{$\alpha=1,\beta=1$}
  \addplot[smooth, samples=100, color=purple]  {betaDist(2,2)};
    \addlegendentry{$\alpha=2,\beta=2$}
  \addplot[smooth, samples=100, color=yellow]  {betaDist(3,3)};
    \addlegendentry{$\alpha=3,\beta=3$}
  \addplot[smooth, samples=100, color=red]     {betaDist(1,2)};
    \addlegendentry{$\alpha=1,\beta=2$}
  \addplot[smooth, samples=100, color=magenta] {betaDist(1,3)};
    \addlegendentry{$\alpha=1,\beta=3$}
  \addplot[smooth, samples=100, color=orange]  {betaDist(1,5)};
    \addlegendentry{$\alpha=1,\beta=8$}
  \addplot[smooth, samples=100, color=cyan]    {betaDist(2,1)};
    \addlegendentry{$\alpha=2,\beta=1$}
  \addplot[smooth, samples=100, color=black]   {betaDist(3,1)};
    \addlegendentry{$\alpha=3,\beta=1$}
  \addplot[smooth, samples=100, color=green]   {betaDist(5,1)};
    \addlegendentry{$\alpha=8,\beta=1$}
  \addplot[smooth, samples=100, color=gray]    {betaDist(3,6)};
    \addlegendentry{$\alpha=3,\beta=6$}
  \end{axis}
\end{tikzpicture}
\caption{Different $\beta$-distributions}
\label{fig:2014-05-30_ex1betadistributions}
\end{figure}

We can easily observe that the $\beta$-distribution\index{B@$\beta$-Distribution} is symmetric for $\alpha=\beta$ and always 1 in case $\alpha=\beta=1$. If we increase one parameter the distribution's peak wanders to one side, towards 0 for higher $\beta$s and towards 1 for higher $\alpha$s.

If we increase both parameters unequally, the distribution's peak moves towards the higher parameter (similar to the movement mentioned before, see $\alpha=3,\beta=6$) and gets for high values much narrower and steeper (not shown in the plot, see figure \ref{fig:2014-05-30_ex3posterior} for an example).

\bigskip

When using the $\beta$-distribution to learn about the probability $q$ for the thumbtack example (page \pageref{example:Thumbtack Toss}) we would start with $\alpha=\beta=1$, i.e. uninformed. After gathering some data we can use it to update our prior belief and adjust the parameters according to our posterior distribution. If we continue doing this iteratively the distribution will get closer to the real $q$ of the thumbtack. % Note: We should find out which distribution we have with n throws and k successes (probably binomial)


\subsection*{Exercise 2}
We have only very limited information and a best guess is that A will win the elections, since there are 53.3\% of the people supporting him. We are not really sure about that because of the lack of information.

Our prior belief is uninformed, i.e. we have no idea about the outcome (Assuming we don't know the poll). The likelihood is our data sample, i.e. the poll's data. It is a binomial distribution\index{Binomial Distribution}: $P(Poll|A) = \binom{1000}{533} p^{533}(1-p)^{1000-533}$. We use the $\beta$-distribution\index{B@$\beta$-Distribution} to model our prior belief: it is conjugate to the Binomial distribution. 

We try to find to find $P(A|Poll)$, i.e. the probability that A wins the election given the poll results.
\begin{align*}
P(A|Poll) &= \frac{P(Poll|A)P(A)}{P(Poll)}\\
P(A|Poll) &= \frac{\binom{1000}{533}p^{533}(1-p)^{467} \cdot p^{\alpha-1}(1-p)^{\beta-1}}{B(\alpha,\beta)}\\
\text{using an uninformed prior, i.e. }\alpha=1, \beta=1\\
P(A|Poll) &= \frac{\binom{1000}{533}p^{533}(1-p)^{467} \cdot p^{1-1}(1-p)^{1-1}}{B(1,1)}\\
P(A|Poll) &\propto \frac{p^{533+1-1}(1-p)^{467+1-1}}{B(534,468)}
\end{align*}

We can plot this posterior belief $P(A|Poll)$ (figure \ref{fig:2014-05-30_ex2plot}) and see how likely it is that A wins. Since the highest density is above 0.5, it's more likely that A wins the election.
\begin{figure}[!ht]
\centering
\begin{tikzpicture}
  \begin{axis}[axis x line*=bottom, axis y line*=left, enlargelimits=upper, axis on top, domain=0:1, legend style={at={(0.98,0.5)}, anchor=east}]
  % csvwrite('2014-05-30_exercise2_pdf.txt',[[0:0.001:1]' betapdf(0:0.001:1,534,468)'])
  \addplot[smooth] file {../data/2014-05-30_exercise2_pdf.txt};
    \addlegendentry{$\genfrac{}{}{0pt}{}{\alpha=534}{\beta=468}$}
  \draw ({axis cs:0.5,0}|-{rel axis cs:0,0}) -- ({axis cs:0.5,0}|-{rel axis cs:0,1});
  \end{axis}
\end{tikzpicture}
\caption{How likely is it, that A wins the election?}
\label{fig:2014-05-30_ex2plot}
\end{figure}


\subsection*{Exercise 3}
We denote $B$ as the number of people believing in UFOs and $D$ as the data (i.e. the poll results) given.

Using the $\beta$-Distribution\index{B@$\beta$-Distribution} we come up with the following model:
\begin{align*}\index{Binomial Distribution}
                P(B|D) &= \frac{P(D|B)P(B)}{P(D)} \\
\Leftrightarrow P(B|D) &= \frac{\binom{500}{230}q^{230}\left(1-q\right)^{500-230}P(B)}{P(D)}\\
                P(B|D) &= \frac{\overbrace{\binom{500}{230}q^{230}\left(1-q\right)^{270}}^\text{Binomial} \overbrace{q^{\alpha-1}\left(1-q\right)^{\beta-1}}^\text{$\beta$-Distribution}}{B(\alpha,\beta)} \\
                P(B|D) &\propto q^{230+1-1}\left(1-q\right)^{270+9-1} \\
                P(B|D) &\Rightarrow \frac{q^{231-1}\left(1-q\right)^{279-1}}{B(231,279)}
\end{align*}

The prior distribution (figure \ref{fig:2014-05-30_ex3prior}) is far to the left, as we have a relatively huge $\beta$ compared to $\alpha$.
\begin{figure}[!ht]
\centering
\begin{tikzpicture}
  \begin{axis}[axis x line*=bottom, axis y line*=left, enlargelimits=upper, axis on top, domain=0:1]
  \plot[smooth, color=blue] file {../data/2014-05-30_priorPDF_data.txt};
    \addlegendentry{$\alpha=1,\beta=9$}
  \end{axis}
\end{tikzpicture}
\caption{Prior $\beta$-Distribution}
\label{fig:2014-05-30_ex3prior}
\end{figure}

With the new $\alpha = 231$ and $\beta = 279$ we can find the $95\%$ interval by calculating the CDF\index{CDF} of the posterior $\beta$-distribution and searching for the x-values for $y_{min}=0.025$ and $y_{max}=0.975$. Using these x-values for the corresponding PDF\index{PDF} yields the $95\%$ interval (figure \ref{fig:2014-05-30_ex3posterior}). The left figure shows the CDF and the intersections for a visual explanation: By using the inverse of the CDF we can calculate the x-values much easier.

\begin{figure}[!ht]
\centering
\begin{tikzpicture}
  \begin{axis}[axis x line*=bottom, axis y line*=left, enlargelimits=upper, axis on top, domain=0:1, name=cdf, legend style={at={(0.98,0.5)}, anchor=east}]
  \plot[smooth, color=blue] file {../data/2014-05-30_posteriorCDF_data.txt};
    \addlegendentry{$\genfrac{}{}{0pt}{}{\alpha=231}{\beta=279}$}
  \plot[color=red] {0.975};
    \addlegendentry{$x=0.975$}
  \plot[color=red] {0.025};
    \addlegendentry{$x=0.025$}
  \node at (axis cs:0.41,0.025) {$\times$};  \node at(axis cs:0.27,0.1) {(0.41|0.025)};
  \node at (axis cs:0.496,0.975) {$\times$}; \node at(axis cs:0.3,0.9) {(0.496|0.975)};
  \end{axis}
  
  \begin{axis}[axis x line*=bottom, axis y line*=left, enlargelimits=upper, axis on top, domain=0:1, name=pdf, at=(cdf.right of east), anchor=left of west]
  \plot+[domain=0.41:0.496, samples=1000, pattern=flexible hatch, mark=none
            hatch distance=0.2pt, hatch thickness=1pt,
            draw=blue, pattern color=cyan, area legend]
            file {../data/2014-05-30_posteriorPDF_area_data.txt} \closedcycle;
    \addlegendentry{$95\%$}
  \plot[smooth, color=red] file {../data/2014-05-30_posteriorPDF_data.txt};
    \addlegendentry{$\genfrac{}{}{0pt}{}{\alpha=231}{\beta=279}$}
  \plot[smooth, color=gray] file {../data/2014-05-30_priorPDF_data.txt};
    \addlegendentry{$\genfrac{}{}{0pt}{}{\alpha=1}{\beta=9}$}
  \end{axis}
\end{tikzpicture}
\caption{Left: Posterior CDF and 95\% intersections. Right: Posterior PDF and 95\% interval, prior PDF for comparison.}
\label{fig:2014-05-30_ex3posterior}
\end{figure}

\bigskip

\texttt{MATLAB} code to try out exercise 3:
\lstinputlisting{../data/sheet5_posteriorBetaDistribution.m}


\subsection*{Exercise 4}
To make a guess we can run a simulation with $n$ subjects who have to do the task 25 times. We measure how often they answer correctly and in the end we check how many people $g$ out of $n$ had more than 20 correct answers.

One possibly result was: \\
\texttt{0.0473\% (473) of 1000000 subjects scored more than 20 out of 25.}

\bigskip

\texttt{MATLAB} code to try out exercise 4:
\lstinputlisting{../data/sheet5_inferenceSimulation.m}


\subsection*{Exercise 5}\index{NHST}
\begin{enumerate}
	\item Since $p < 0.01$ it's not absolutely disproved, because we still have to take $\beta$ into account.
  \item You don't find the probability of the $H_0$ ($p(q=\frac{1}{2}|H=h,n=25)$), but instead you find $p(H>20|q=\frac{1}{2},n=25)$ - the p-value. But the p-value is not sufficient to know the probability of $H_0$.
  \item As in 1: it's not certain.
  \item Finding $p(q>\frac{1}{2}|H=h,n=25)$ is only possible with Bayes' rule\index{Bayes' Rule}, so it's impossible from a frequentist point of view.
  \item To know the probability that the decision taken was wrong you need to know the type II ($\beta$-) error. This is not possible if you don't know $q$, which you don't in an NHST.
  \item It's not possible to know what happens if you repeat an experiment. The results might be similar or totally different - so they are definitely not significant with a probability of 99\%.
  \item This is the only true fact in this list that we can derive from an NHST. % TODO: Why?
\end{enumerate}


\subsection*{Exercise 6}\index{NHST}
We want to find out the relative proportion of true hypotheses to hypotheses which were wrongly found out to be true, i.e. $\frac{\text{true hypotheses}}{\text{hypotheses found true}}$.

By assuming that the probability for a hypotheses to be true is $q$, we can come up with a simple probability tree (figure \ref{fig:2014-05-30_ex6tree}).

\index{Probability tree}
\begin{figure}[!ht]
  \tikzstyle{level 1}=[level distance=3cm, sibling distance=4.5cm]
\tikzstyle{level 2}=[level distance=3cm, sibling distance=2.5cm]
\tikzstyle{level 3}=[level distance=3cm, sibling distance=2.0cm]

\tikzstyle{bag} = [text width=4em, text centered]
\tikzstyle{end} = [circle, minimum width=3pt, fill, inner sep=0pt]

\begin{tikzpicture}[grow=right, sloped]
\node[bag] {all hypotheses}
    child {
        node[bag] {true}
        child {
            node[end, label=right: {NHST false: $\beta q$}] {}
            edge from parent
            node[above] {}
            node[below] {$\beta$}
        }
        child {
            node[end, label=right: {NHST true: $(1-\beta)q$}] {}
            edge from parent
            node[above] {$1-\beta$}
            node[below] {}
        }
        edge from parent         
        node[above] {}
        node[below] {$q$}
    }
    child {
        node[bag] {false}
        child {
            node[end, label=right: {NHST false: $(1-\alpha)(1-q)$}] {}
            edge from parent
            node[above] {}
            node[below] {$1-\alpha$}
        }
        child {
            node[end, label=right: {NHST true: $\alpha(1-q)$}] {}
            edge from parent
            node[above] {$\alpha$}
            node[below] {}
        }
        edge from parent         
        node[above] {$1-q$}
        node[below] {}
    };
\end{tikzpicture}

  \caption{Probability tree}
  \label{fig:2014-05-30_ex6tree}
\end{figure}

By using the formulas for hypotheses which were found to be true we can fill the formula mentioned above.
\begin{align*}
\frac{\overbrace{(1-\beta)q}^\text{true hypotheses}}{\underbrace{(1-\beta)q+\alpha(1-q)}_\text{hypotheses found true}}
\end{align*}
We can solve this for $q$ by plugging in the given values $\alpha=5\%,1-\beta=75\%$ and setting it to be smaller than $\frac{1}{2}$.
\begin{align*}
\frac{1}{2}&>\frac{(1-\beta)q}{(1-\beta)q+\alpha(1-q)} \\
\frac{1}{2}&>\frac{0.75q}{0.75q+0.05(1-q)} \\
\frac{1}{2}&>\frac{0.75q}{0.75q+0.05-0.05q} \\
\frac{1}{2}&>\frac{0.75q}{0.7q+0.05} \\
0.7q+0.05  &>1.5q \\
0.05       &>0.8q \\
0.0625     &>q
\end{align*}
So if $q$ is smaller than $\frac{1}{16}$, the probability that the effect is real is less than 50\%.


\subsection*{Exercise 8}
This exercise was posed again later, since it was not crucial for the midterm exam. For the solution please refer to page \ref{sheet6ex2}.

\end{document}